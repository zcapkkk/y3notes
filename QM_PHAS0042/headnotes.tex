% Created 2021-11-23 Tue 17:04
% Intended LaTeX compiler: pdflatex
\documentclass[11pt]{article}
\usepackage[utf8]{inputenc}
\usepackage[T1]{fontenc}
\usepackage{graphicx}
\usepackage{grffile}
\usepackage{longtable}
\usepackage{wrapfig}
\usepackage{rotating}
\usepackage[normalem]{ulem}
\usepackage{amsmath}
\usepackage{textcomp}
\usepackage{amssymb}
\usepackage{capt-of}
\usepackage{hyperref}
\author{Kelvin Ho}
\date{\today}
\title{Quantum Mechanics Head notes}
\hypersetup{
 pdfauthor={Kelvin Ho},
 pdftitle={Quantum Mechanics Head notes},
 pdfkeywords={},
 pdfsubject={},
 pdfcreator={Emacs 26.3 (Org mode 9.1.9)}, 
 pdflang={English}}
\begin{document}

\maketitle
\tableofcontents

Notes to jog my memory.

\section{Heisenberg Matrix Approach}
\label{sec:orgbe8c3d0}

Add notes to talk about some rules on linear algebra generalisation of
Quantum Mechanics.

\section{Algebraic Approach to Harmonic Oscillator}
\label{sec:org12c6c1a}

Add notes to talk about the raising and lowering operators

\section{Generalised Angular Momentum}
\label{sec:org99167db}

\subsection{Preliminaries}
\label{sec:orge2323e0}

Begin with:

\[ \left[ \hat{J}_x, \hat{J}_y \right] = i\hbar \hat{J}_z \]

Other pairing can be achieved through cyclic permutations.

The raising and lowering operators are:

\[ \hat{J}_\pm = \hat{J}_x \pm i\hat{J}_y \]

They are Hermitian conjugates of each other. 

The product is defined by:

\[ \hat{J}_\pm\hat{J}_\mp = \hat{J}^2 - \hat{J}^2_z \pm \hbar\hat{J}_z\]
(The last sign follows the first.)

Thus, the commutator is:
\[ \left[ \hat{J}_\pm,\hat{J}_\mp\right] = \pm 2\hbar\hat{J}_z\]
(Again, the sign follows the leader.)

The raising/lowering operators commutes with \(\hat{J}^2\), but with the
components:

\[ \left[ \hat{J}_z, \hat{J}_\pm \right] = \pm \hbar\hat{J}_\pm \]
(Follows the sign.)

\subsection{Eigenstates and Eigenvalues}
\label{sec:org040af85}

We begin again by stating that:
\begin{align*}
\hat{J}^2 |\phi\rangle &= \alpha |\phi\rangle\\
\hat{J}_z |\phi\rangle &= \beta |\phi\rangle
\end{align*}

The raising/lowering operators raises or lowers \(\hat{J}_z\) in \(\hbar\)
increments.

However, as it commutes with \(\hat{J}^2\), it does nothing. 

Since \(\alpha\) is fixed and by definition bigger than \(\hat{J}_z\) (as
it is a component), it follows that there is a termination
i.e. bounded upper and lower eigenvalues.

\subsubsection{Proof}
\label{sec:orgdc42ac6}

To show, write a top eigenstate where applying the raising operator
would yield 0, and vice versa for the lowest state. The difference in
eigenvalues is \(n\hbar\). Using the commutator of the raising and
lowering operator, this can be easily shown.

From:
\[m_T - m_B = n\hbar\]
Then:
\[\alpha = mT(mT +\hbar)\]
And since:
\[ m_T = n\frac{\hbar}{2}\]
Rewriting \(j = \frac{n}{2}\):
\begin{align*}
\alpha &= j(j+1)\hbar^2\\
\beta &= m\hbar
\end{align*}

\subsection{Actions of the Operators}
\label{sec:org4a18614}

Rewrite \(|\phi\rangle = |j, m\rangle\) The raising operator gives:
\(\hat{J}_z\left(\hat{J}_+|j,m\rangle\right) =
(m+1)\hbar\left(\hat{J}_+|j,m\rangle\right)\), but applying \(\hat{J}_z\)
to \(|j,m+1\rangle\) yields the same eigenvalue(?), implying they are
proportional to each other. With this, we write: \[
\hat{J}_+ |j,m\rangle = C_+ |j, m+1\rangle \] Rewriting the above as a
bra and noticing that the raising/lowering operators are Hermitian
adjoint, using their product yields (usual convention takes positive root):
\begin{align*}
C_+ &= \hbar\sqrt{j(j+1)-m^2-m}\\
C_- &= \hbar\sqrt{j(j+1)-m^2+m}\\
\therefore C_\pm &= \hbar\sqrt{j(j+1)-m(m\pm 1)}
\end{align*}
(Again, m sign follows leading sign)

\section{Spin (Spin \textonehalf{} systems)}
\label{sec:orgda054f5}

The spin states are a carbion copy of the algebraic theory of
generalised angular momentum (Griffith). They do not have an
equivalent position representation. 


\subsection{Preliminaries}
\label{sec:org7459ef0}

Spin up and spin down states:

\begin{align*}
 | \frac12, \frac12 \rangle &\equiv |\alpha\rangle  = |+\rangle\\
 | \frac12, -\frac12 \rangle &\equiv |\beta \rangle = |-\rangle
\end{align*}

Again:
\begin{align*}
\hat{S}^2 |+\rangle &= s(s+1)\hbar^2|+\rangle = \frac34\hbar^2|+\rangle\\
\hat{S}_z |+\rangle &= s\hbar|+\rangle = \frac12|+\rangle\\
\hat{S}^2 |-\rangle &= s(s+1)\hbar^2|-\rangle = \frac34\hbar^2|-\rangle\\
\hat{S}_z |-\rangle &= s\hbar|-\rangle = -\frac12|-\rangle\\
\end{align*}

\subsection{Raising and Lowering Operators}
\label{sec:orgfadefcb}

Same as before:
\begin{align*}
\hat{S}_\pm|s,m\rangle &= \hbar\sqrt{s(s+1)-m(m\pm1)}|s,m\pm1\rangle\\
\therefore \hat{S}_\pm|\frac12,m\rangle &= \hbar\sqrt{\frac34-m(m\pm1)}|\frac12,m\pm1\rangle\\
\end{align*}

Thus: 
\begin{itemize}
\item Raising spin down becomes \(\hbar\) times spin up
\item Lowering spin up becomes \(\hbar\) times spin down
\item Anything else is zero (cannot raise above the ceiling or go below
the floor)
\end{itemize}

This might come in handy: \[ \hat{S}_x = \frac12 \left(\hat{S}_+ +
\hat{S}_-\right); \qquad \hat{S}_y =
\frac{1}{2i}\left(\hat{S}_+-\hat{S}_-\right)\]

\subsubsection{The arbitrary state}
\label{sec:orgb596c86}

An arbitrary state can be written as:
\[ |\chi \rangle = a|+\rangle + b|-\rangle \]


\subsubsection{Additional Relations}
\label{sec:org5569682}

The \(\hat{S}^2\) is a purely numerical operator for any state:
\[ \hat{S}^2 \equiv \frac34 \hbar^2\]

Additionally, for spin-\textonehalf{} states:
\[ \hat{S}^2_x = \hat{S}^2_y = \hat{S}^2_x = \frac{\hbar^2}{4} \]

\begin{enumerate}
\item Anticommutators:
\label{sec:orgf979bb2}

\[ \left\{\hat{S}_x, \hat{S}_y\right\} = 0 \]
\end{enumerate}

\subsubsection{Matrix Representation}
\label{sec:org7d65110}

This section introduces the Pauli matrices. These are formed by
performing: \(\langle s',m'| \hat{S} |s,m \rangle\) for each respective
operators. Each operator, \(\hat{S}^2, \hat{S}_x, \hat{S}_y, \hat{S}_z\)
can be formed by multiplying their respective matrices with \(\hbar/2\):

\begin{align*}

I &= 

\begin{pmatrix}
1 & 0 \\
0 & 1\\
\end{pmatrix}\\

\sigma_x = 
\begin{pmatrix}
0 & 1 \\
1 & 0\\
\end{pmatrix}\\

\sigma_y =
\begin{pmatrix}
0 & -i \\
i & 0\\
\end{pmatrix}\\

\sigma_z = 
\begin{pmatrix}
1 & 0 \\
0 & -1 \\
\end{pmatrix}

\end{align*}

The trace of the Pauli matrices are zero, and the determinant is -1. 

\subsection{Determination of eigenstates and eigenvectors}
\label{sec:org5a1eb00}

To solve for \(\hat{S}_x |\chi\rangle = a |\chi \rangle\), we write \(a =
\frac\hbar2\lambda\) and \(\hat{S}_x = \frac\hbar2\sigma_x\), and let
\(|\chi\rangle = (u,v)^T\). Thus the matrix approach gives an easy way
to determine the eigenstates and eigenvalues of the operators.

Equivalently, this can be performed via the basis states and
raising/lowering operators approach.

See page 54 of notes for a table showing the results. 


\subsection{Spin along arbitrary direction}
\label{sec:orgece081d}

The operator \(\hat{S}_n\) is defined as \(\hat{S}\cdot\hat{n}\), where:
\[ \hat{n} = (\cos\phi\sin\theta, \sin\phi\sin\theta,\cos\theta) \]

Thus: 
\begin{equation*} \hat{S}_n = \frac{\hbar}{2}(\hat{\sigma}_x
\cos{\phi}\sin{\theta}+\hat{\sigma}_y\sin{\phi}\sin{\theta}+\hat{\sigma}_z\cos{\theta})
= \frac{\hbar}{2}
\begin{pmatrix}
\cos\theta & \sin\theta e^{-i\theta} \\
\sin\theta e^{i\theta} & -\cos\theta
\end{pmatrix} 
\end{equation*}

Then, using trigonometric identities (half angle?):

\begin{equation*}
|+\rangle_\hat{n} = 
\begin{pmatrix}
\cos{\frac{\theta}{2}}\\
\sin{\frac{\theta}{2}}e^{i\phi}
\end{pmatrix} \qquad

|-\rangle_\hat{n} = 
\begin{pmatrix}
\sin{\frac{\theta}{2}}\\
-\cos{\frac{\theta}{2}}e^{i\phi}
\end{pmatrix}


\end{equation*}


\begin{itemize}
\item Spin up along \(+z\) is \(P_+ = cos^2{\frac\theta2}\)
\item Spin down is \(P_- = sin^2{\frac\theta2}\)
\end{itemize}



\subsection{Addition of Angular Momentum}
\label{sec:orgf726f1f}

In the special case of spin-\textonehalf{} particles, the addition of angular
momentum is simple:

\(\hat{\mathbb{S}} = \hat{\mathbb{S_1}} + \hat{\mathbb{S_2}}\)

There are four possible states for the 2 particles:
\[ |++\rangle, |+-\rangle, |-+\rangle, |--\rangle\]

Then, from above:
\[ \hat{S}_z = \hat{S}_{1z} + \hat{S}_{2z}\]
Eigenvalues are thus: 1,0,0,-1.


\begin{center}
\begin{tabular}{lll}
1$\backslash$2 & \(\alpha\) & \(\beta\)\\
\(\alpha\) & \(\alpha_{\text{1}}\) \(\alpha_{\text{2}}\) & \(\alpha_{\text{1}}\) \(\beta_{\text{1}}\)\\
\(\beta\) & \(\beta_{\text{1}}\) \(\alpha_{\text{2}}\) & \(\beta_{\text{1}}\) \(\beta_{\text{2}}\)\\
\end{tabular}
\end{center}

Denote a general state by \(|S, M\rangle\). 

If \(M = \frac12+\frac12 = 1\) must have \(S=1\), as: \[
\hat{S}_z|1,1\rangle =
\left(\hat{S}_{1z}+\hat{S}_{2z}\right)\alpha_1\alpha_2 =
\frac\hbar2\alpha_1\alpha_2+\frac\hbar2\alpha_1\alpha_2 =
\hbar\alpha_1\alpha_2\]

\section{Total Angular Momenta}
\label{sec:orgf8583ee}

We define:
\[ \hat{J} = \hat{L}+\hat{S}\]

While Y\(_{\text{lm}} \alpha\) and Y\(_{\text{lm}} \beta\) are eigenstates of \(\hat{J}_z\),
they are not eigenstates of \(\hat{J}^2\).

\subsection{Example}
\label{sec:orgddcb413}

Let J = l+\textonehalf{} and m\(_{\text{j}}\) = l+\textonehalf{} i.e. the state of maximum projection Y\(_{\text{ll}\alpha}\) \(\equiv\) |j,j\(\rangle\).
Applying the lowering operator \(\hat{J}_-|j,m\rangle = \sqrt{j(j+1)-m(m-1)}\hbar|j,m-1\rangle\):

\begin{align*}
\hat{J}_- |j,j\rangle &= \sqrt{2\hbar}|j,j-1\rangle \\
\hat{J}_- |l+\frac12, l+\frac12\rangle &= \sqrt{2\left(l+\frac12\right)}\hbar|l+\frac12,l-\frac12\rangle\\

\left(\hat{L}_- + \hat{S}_-\right)Y_{ll}\alpha = \left(\hat{L}_- +
\hat{S}_-}\right)|l,l\rangle|\alpha\rangle &=
\sqrt{2l}\hbar|l,l-1\rangle|\alpha\rangle + |l,l\rangle\hbar|\beta\rangle\\
\end{align*}
Equating the two and using orthogonality conditions: 
\begin{align*}
\sqrt{2\left(l+\frac12\right)}\hbar|l+\frac12,l-\frac12\rangle &=
\sqrt{2l}\hbar|l,l-1\rangle|\alpha\rangle + |l,l\rangle\hbar|\beta\rangle
|l+\frac12, l-\frac12\rangle &=
\sqrt{\frac{2l}{2l+1}|l,l-1\rangle|\alpha\rangle+\sqrt{\frac{1}{2l+1}|l,l\rangle|\beta\rangle\\
\end{align*}

Introduce a new state \(|l-\frac12,l-\frac12\rangle =
a|l,l-1\rangle|\alpha\rangle+b|l,l\rangle|\beta\rangle\) which is
orthogonal:
\end{document}
