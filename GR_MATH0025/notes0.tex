% Created 2021-10-21 Thu 23:21
% Intended LaTeX compiler: pdflatex
\documentclass[11pt]{article}
\usepackage[utf8]{inputenc}
\usepackage[T1]{fontenc}
\usepackage{graphicx}
\usepackage{grffile}
\usepackage{longtable}
\usepackage{wrapfig}
\usepackage{rotating}
\usepackage[normalem]{ulem}
\usepackage{amsmath}
\usepackage{textcomp}
\usepackage{amssymb}
\usepackage{capt-of}
\usepackage{hyperref}
\author{Kelvin}
\date{\today}
\title{Overall Notes for General Relativity}
\hypersetup{
 pdfauthor={Kelvin},
 pdftitle={Overall Notes for General Relativity},
 pdfkeywords={},
 pdfsubject={},
 pdfcreator={Emacs 26.3 (Org mode 9.1.9)}, 
 pdflang={English}}
\begin{document}

\maketitle
\tableofcontents



\section{Week 1}
\label{sec:org433dd05}
\subsection{Vector that lives on the sheet}
\label{sec:orgcc7c86b}

\(f: V \to R\)

\begin{itemize}
\item Parameterization that forces vector to live inside a curved surface
\begin{itemize}
\item Recall from Newtonian mechanics
\end{itemize}
\item a bunch of stuff I have to go through again 
\begin{itemize}
\item Covariant and contravariant vectors
\begin{itemize}
\item Derivation
\end{itemize}
\item Coordinate transformation
\end{itemize}
\end{itemize}

\section{Week 2}
\label{sec:org0f1d8da}
\subsection{Tensors}
\label{sec:org445a5b6}
Tensor of type (p,q) is: \(T^{i_1,i_2,...,i_p}_{j_1,j_2,...,j_q}\)
\subsubsection{Tensors have upper and lower indices:}
\label{sec:org464ca64}
\begin{itemize}
\item Upper indices transform contravariantly
\item Lower indices transform covariantly
\end{itemize}
\subsubsection{Ranks}
\label{sec:orgba3cfe7}
\begin{itemize}
\item Scalar: 0
\item Vectors: 1
\item We will not go beyond rank 4 tensors in this course
\end{itemize}
\subsubsection{Contraction}
\label{sec:orgea361d6}
\begin{itemize}
\item Summing over a pair of indices results in a tensor of rank = p+q-2
\item Proof requires showing that the a set of Jacobian cancels out
\item This forms a kronecker delta, which does not change transformation
\item QED
\end{itemize}
\subsubsection{Partial Derivatives}
\label{sec:orgdfb5430}
\begin{itemize}
\item Not a tensor
\item Transformation properties of PD
\item Rewriting it as a chain rule and product rule
\item One term obeys tensor trans. but the other does not
\end{itemize}
\subsubsection{Manifolds}
\label{sec:org88d1bd0}
\begin{itemize}
\item An n dimensional, smooth (inf. diff.) is a set
\item with collection of open subsets \({\omega_\alpha}\)
\item a collection of coordinates \({X^i_{(\alpha)}}\)
\item Where \(\alpha \in N\) and \(i = 1,2,...,n\)
\begin{itemize}
\item every point on the manifold belongs to at least one \$\{\(\Omega_{\alpha}\)\}
\end{itemize}
\item EXAMPLE: Surface of Earth is a manifold
\item Collection are pages of atlas
\item Overlap
\end{itemize}
\subsubsection{Distance}
\label{sec:org25aa65d}
\begin{itemize}
\item \(g_{ij}\) is a rank 2 tensor known as metric tensor
\item Represents distance
\item Inverse: \(g^{jk}\)
\item The line element: \$ds\(^{\text{2}}\) = g\(_{\text{ij}}\)dX\(^{\text{idX}}\)\(^{\text{j}}\)
\begin{itemize}
\item Where \(g_{ij}\) are arbitrary function of coordinates
\item If it is I, then it becomes \(ds^2 = dx^2+dy^2+dz^2\)
\end{itemize}
\end{itemize}
Remark:
\begin{center}
\begin{tabular}{rr}
Dimensions & Independent components\\
\hline
2D & 3\\
3D & 6\\
4D & 10\\
N-D & \(\frac{N(N+1)}{2}\)\\
\end{tabular}
\end{center}
\begin{itemize}
\item Other motivation for the metric structure:
\end{itemize}
To introduce a mapping that links the covariant and the contravariant
\subsubsection{Metric Signature}
\label{sec:org56dc25d}
\begin{itemize}
\item Counts the number of positive and negative eigenvalues
\item Kronecker delta: (/+,/+,+)
\item GR: (-,/+,+,+)
\item Line elements or metric in GR are called Lorentzian manifolds
\end{itemize}
\subsubsection{Geodesic Equation}
\label{sec:org7b14002}
\begin{itemize}
\item Def: A curve \(C\) given by \(X^i(\lambda)\) is \textbf{geodesic}
\item if it satisfies \(\ddot(X)^n+Gamma^{n}_{ms}\dot(X)^m\dot(X)^s = 0\)
\end{itemize}
Where \$\(\Gamma^{\text{n}}_{\text{ms}}\) = \textonehalf{} g\(^{\text{nk}}\)\eft(\(\partial_{\text{m}}\) g\(_{\text{sk}}\) + \(\partial_{\text{s}}\) g\(_{\text{km}}\) - \(\partial_{\text{k}}\) g\(_{\text{ms}}\)$\backslash$
\begin{itemize}
\item Geodesics are shortest distances between points
\item Get metric tensors and you can computed
\item \textbf{Christoffel Symbol}: \(\Gamma^n_{ms}\)
\item Derive from E-L and learn the actual derivations, watch it a few times
\end{itemize}
\subsection{Summary:}
\label{sec:orga388aff}
\section{Week 3}
\label{sec:orga162787}
\subsubsection{Geodesic Equation in Polar Coordinates}
\label{sec:org6299e36}
\begin{itemize}
\item Solutions are given by straight lines (Exercise is to show this)
\end{itemize}
\subsubsection{The Christoffel symbol can be found in two ways}
\label{sec:orgadcde4e}
\begin{itemize}
\item Explicit formula
\item via Lagrangian approach
\end{itemize}
\subsubsection{Simplified/reduced Lagragian gives the same EL equations}
\label{sec:org6508df0}
\begin{itemize}
\item However one is used over the other
\item Because sqrt is motivated from arc
\end{itemize}
\subsubsection{Metric tensor}
\label{sec:orgf0d3028}
\begin{itemize}
\item \(g_{ij}T^iT^j = ||T||^2\)
\item Can be used to raise or lower indices (see Zee)
\item Upper is column, lower is row
\begin{enumerate}
\item \(T_i^j = g_{is}T^{sj}\)
\item \(T_j^i = g_{js}T^{is}\)
\item \(T_{ij} = g_{im}g_{jn}T^{mn}\)
\end{enumerate}
\end{itemize}
\subsubsection{Covariant Derivative}
\label{sec:orgcc46459}
\begin{itemize}
\item A covariant derivative \(\nabla_a\) on a mapping
\item From tensors of type (p,q) to tensors type (p,q+1)
\item Has the following properties:
\begin{enumerate}
\item If f is a smooth function, then \(\nabla_{a}f=\partial_{a}f\)
\item \(\nabla_a\) is linear (like Operators)
\item Leibniz (product) rule
\item Commutes with contraction
\begin{itemize}
\item Contract first then differentiate or vice versa
\end{itemize}
\end{enumerate}
\end{itemize}
\subsubsection{Metricity}
\label{sec:org814dd78}
\begin{itemize}
\item Something is metric compatitable if:
\item \(\nabla_a g_{bc} = 0\)
\item Else, the non-metricity is \(Q_abc =  \nabla_a g_{bc} = 0\)
\end{itemize}
\subsubsection{Torsion}
\label{sec:org9355679}
\begin{itemize}
\item Torsion is defined by:
\item \((\nabla_a\nabla_b f - \nabla_b \nabla)a f) = T_{ab}^c\partial_{cf}\)
\item GR assumes \(T_{ab}^c=0\)
\end{itemize}
\subsubsection{Fine \(\nabla_a A^i\)}
\label{sec:org831b1eb}
\begin{itemize}
\item \(\partial_a A^i\) is not a rank 2 tensor
\item \(\partial_a f = \nabla_a A^i\)
\item Thus \(\partial_a A^i\) should be linear in \(A^i\)
\item Something about showing a C from the equation of this function
\end{itemize}
\subsubsection{Christoffel symbol components}
\label{sec:org4310a85}
\begin{itemize}
\item The C are actually the Christoffel symbol
\begin{itemize}
\item Show that it is symmetric
\item Torsion free
\item Examinable proof of writing out the three equations
\begin{itemize}
\item Adding up and subtracting
\end{itemize}
\end{itemize}
\end{itemize}
\subsubsection{Parallel Transport}
\label{sec:orga16b15f}
\begin{itemize}
\item In GR, the straightest curves are shortest
\item However, in presence of torsion, this is no longer the case
\end{itemize}
\end{document}
